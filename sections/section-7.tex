\section{语义分析和中间代码生成}

这章内容更少,主要是不知道考什么,也没讲什么。

\subsection{中间语言}
\subsubsection{后缀式}

后缀表示法又称逆波兰表示法,这种表示法是把操作数写在前面,把算符写在后面。一个表达式 E 的后缀形式定义如下:
\begin{itemize}
    \item 如果 R 是一个变量或常量,其后缀式是本身。
    \item 如果 E 是 $E_1 op E_2$ 的形式,则变成 $E_1'E_2'op$ 这里 $E_1',E_2'$ 表示原先的后缀式。
    \item (E) 的后缀式和 E 的后缀式相同。
\end{itemize}

后缀表达式转换学过很多遍了,不再赘述,参考文献: \url{https://blog.csdn.net/qq_36631580/article/details/88685588}

\subsubsection{DAG 图}
DAG(无循环有向)图和之前构建的语法树非常相似,不同的是一个节点可以拥有几个父节点。没了。

比如有个表达式: $a+a*(b-c)+(b-c)*d$

\begin{figure}[H]
    \centering
    \begin{tikzpicture}[font = \small,thick]
        \node (1-1) at (0,0) {+};
        \node (2-1) at (-1.5,-1.5) {+};
        \node (2-2) at (1.5,-1.5) {*};
        \node (3-1) at (-0.5,-3) {*};
        \node (3-2) at (3,-3) {d};
        \node (4-1) at (-1.5,-4.5) {a};
        \node (4-2) at (0.5,-4.5) {-};
        \node (5-1) at (0,-6) {b};
        \node (5-2) at (1.5,-6) {c};
        \draw (1-1)--(2-1) (1-1)--(2-2) (2-1)--(3-1) (2-2)--(4-2) (2-2)--(3-2) (3-1)--(4-1) (3-1)--(4-2) (4-2)--(5-1) (4-2)--(5-2);
        \draw (2-1) to[out=210,in=150] (4-1);
    \end{tikzpicture}
    \caption{DAG}
    \label{fig:DAG}
\end{figure}

\subsubsection{三地址代码}
为什么不问问许小龙呢?

\subsection{总结与题型}

\includegraphics[width=16cm]{../figures/7.中间代码生成.pdf}

\includegraphics[width=16cm]{../figures/题目DAG的生成与优化.pdf}